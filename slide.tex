\documentclass[a4j,dvipdfmx]{beamer}
\usepackage{pxrubrica}
\title{数値解析が\jruby{乱数}{み|だ}れる}
\author{13024156 藤原 渓亮}
\begin{document}
  \maketitle
  \begin{frame}{今日やること}
    数値解析と乱数の関係を紹介していきます.
  \end{frame}
  \begin{frame}{数値解析とは}
    代数学的に解けない解析学の問題に対して代数式を用いて近似的に解を得る学問.
    \begin{center}$\Downarrow$\end{center}
    得られた解は単純な数値のみで扱う
    \begin{center}$\Downarrow$\end{center}
    数値のみで扱うということは計算機のアルゴリズムで記述可能
  \end{frame}
  \begin{frame}
    \begin{itemize}
      \item 代数方程式
      \item 逆行列
      \item 微分方程式
      \item 積分,重積分
      \item etc
    \end{itemize}

  \end{frame}
  \begin{frame}{乱数(列)とは}
    出力が一意的ではない数字のこと,法則性がない数列のこと
    \begin{itemize}
      \item
    \end{itemize}
  \end{frame}
\end{document}
