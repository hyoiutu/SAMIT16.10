\documentclass[dvipdfmx]{standalone}

\usepackage{color}
\usepackage{xcolor}
\usepackage{tikz}
\usepackage{pgfplots}
\usepackage{ifthen}

\usetikzlibrary{calc}
\pgfmathsetseed{20}

\newcounter{begin}
\newcounter{end}

\newcounter{numofplot}
\newcounter{redplot}
\newcounter{blueplot}

\setcounter{begin}{0}
\setcounter{end}{4}

\setcounter{numofplot}{0}
\setcounter{redplot}{0}
\setcounter{blueplot}{0}

\newcount\input

\message{プロット点数を入力してください}
\read-1 to \input

\begin{document}

  \begin{tikzpicture}[declare function = {func(\x) = \x;}]
    \draw [ultra thick] (\arabic{begin},\arabic{begin}) rectangle (\arabic{end},\arabic{end});
    \draw[thick]  plot[domain=\arabic{begin}:\arabic{end},samples=100] (\x,{func(\x)});
    \foreach \i in {1,...,\input}
    {
      \pgfmathsetmacro\myX{random(\arabic{begin},\arabic{end}-1) + random(0,1000)/1000}%
      \pgfmathsetmacro\myY{random(\arabic{begin},\arabic{end}-1) + random(0,1000)/1000}%
      \pgfmathsetmacro{\rndcolor}{ func(\myX)<\myY ? "red" : "blue" }
      \pgfmathparse{func(\myX)<\myY ? "red" : "blue"}
      \ifthenelse{\equal{\pgfmathresult}{red}}{\stepcounter{redplot}}{\stepcounter{blueplot}}
      \filldraw[fill=\rndcolor] (\myX, \myY) circle [radius=(\arabic{end}-\arabic{begin})*0.01];
      \setcounter{numofplot}{\i}
    }
    \pgfmathparse{(\arabic{end}-\arabic{begin})^2}
    \def\overallS{\pgfmathresult}
    \pgfmathsetmacro\result{(\arabic{blueplot}/\arabic{numofplot})*\overallS/4}

    \draw (7,5) node {全体の点数 = \arabic{numofplot}};
    \draw (7,4) node {青い点数 = \arabic{blueplot}};
    \draw (7,3) node {赤い点数 = \arabic{redplot}};
    \draw (7,2) node {全体的な面積 = \overallS};
    \def\mathfunction{$\int f(x)dx$}
    \draw (7,1) node {\mathfunction = \pgfmathparse{\result}\pgfmathresult};
  \end{tikzpicture}
\end{document}
