\documentclass[dvipdfmx]{standalone}

\usepackage{color}
\usepackage{xcolor}
\usepackage{tikz}
\usepackage{pgfplots}
\usepackage{ifthen}

\usetikzlibrary{calc}
\pgfmathsetseed{20}

\newcount\tmpinput

\newcount\Begin
\newcount\End

\newcount\numofplot
\newcount\redplot
\newcount\blueplot

\begin{document}

  \message{積分範囲の下限を入力してください}
  \read-1 to \tmpinput
  \Begin = \tmpinput
  \message{積分範囲の上限を入力してください}
  \read-1 to \tmpinput
  \End = \tmpinput

  \begin{tikzpicture}[declare function = {func(\x) = exp(\x);}]
    %\draw [ultra thick] (\Begin,{func(\Begin)}) rectangle (\End,{func(\End)});
    \draw[thick]  plot[domain=\Begin:\End,samples=100] (\x,{func(\x)});
    \foreach \i in {1,...,50}
    {
      \pgfmathsetmacro\myX{random(\Begin,\End-1) + random(0,1000)/1000}%
      \pgfmathsetmacro\myY{random({func(\Begin)},{func(\End-1)}) + random(0,1000)/1000}%
      \pgfmathsetmacro{\rndcolor}{ func(\myX)<\myY ? "red" : "blue" }
      \pgfmathparse{func(\myX)<\myY ? "red" : "blue"}
      \ifthenelse{\equal{\pgfmathresult}{red}}{\advance \redplot by 1}{\advance \blueplot by 1}
      \filldraw[fill=\rndcolor] (\myX, \myY) circle [radius=(\End-\Begin)*0.01];
    }
    \numofplot = \i
    \pgfmathsetmacro\result{(\End-\Begin)*func(\End)*\blueplot/\numofplot}
    \pgfmathparse{(\End-\Begin)*func(\End)}
    \def\overallS{\pgfmathresult}
    \draw (7,5) node {全体の点数 = \the\numofplot};
    \draw (7,4) node {青い点数 = \the\blueplot};
    \draw (7,3) node {赤い点数 = \the\redplot};
    \draw (7,2) node {全体的な面積 = \overallS};
    \def\mathfunction{$\int  f(x)dx$}
    \draw (7,1) node {\mathfunction = \pgfmathparse{\result}\pgfmathresult};
  \end{tikzpicture}
\end{document}
